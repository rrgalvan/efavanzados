%==================================================
\section{Elementos finitos de orden alto}
%==================================================

\begin{contenidos}
  Elementos finitos de orden alto. Bases de Lobatto. Bases de
  Bernstein. Teoría e implementación en el ordenador.
\end{contenidos}


Para esta sección, seguiremos el TFG «Reesolución Numérica de EDP
Mediante Elementos Finitos de Orden Alto y Programaciónn en Paralelo»,
presentado por Estefanía Alonso Alonso en el Departamento de
Matemáticas de la Universidad de Cádiz (Julio 2016). El PDF está
\href{https://rodin.uca.es/xmlui/bitstream/handle/10498/19255/TFG-EstefaniaAlonso.pdf}{disponible
  en el respositorio de institucional de la UCA}.

Para profundizar en el asunto, se recomienda el libro de P. \v{S}olin~\cite{solin_higher-order_2004}.

La estructura del TFG antes referido es:
\begin{itemize}

\item \textit{Capítulo 1}: Introducción a la formulación variacional de EDP.
  Se introduce el método de Galerkin y para ello se realiza una breve
  aproximación a los espacios de Sobolev.
\item \textit{Capítulo 2}: Elementos finitos nodales. Es establecen
  las bases para el estudio del método de los elementos finitos.  Se
  plantean de forma rigurosa (en el «sentido de Ciarlet») las
  definiciones y los resultados matemáticos que son de vital
  importancia para su posterior aplicación. Además, de define la
  familia de elementos “clásica” conocida como elementos (nodales) de
  Lagrange.
\item \textit{Capítulo 3}: Polinomios ortogonales y elementos finitos
  jerárquicos.  Se introducen algunas familias de elementos finitos
  que mantienen buenas propiedades cuando aumenta el orden de los
  polinomios.

 \end{itemize}

%%% Local Variables:
%%% mode: latex
%%% TeX-master: "efavanzados.tex"
%%% End:
