
%==================================================
\section{Introducción}
%==================================================

\begin{contenidos}
  Introducción a la «shell» en entornos Unix. Sistemas
  de control de versiones. Lenguaje de programación
  Python. Introducción al entorno de elementos finitos
  FEniCS.
\end{contenidos}

\subsection{Introducción a la «shell» en entornos Unix}
\label{sec:intr-la-shell}

El objetivo de esta sección es familiarizarse con: algunos de los
conceptos básicos de la \textit{shell} UNIX, que serán útiles para el
resto de los temas. En concreto:
\begin{itemize}
\item Órdenes básicas de la \textit{shell}
\item Introducción a la programación de \textit{scripts}
\end{itemize}

Nos centramos en la \textit{shell Bash}, la más
extendida en entornos UNIX. Como punto de partida, se puede utilizar
alguna de las numerosas guías o «reference cards» disponibles en
internet, por ejemplo:
\begin{itemize}
\item \href{https://www.cs.jhu.edu/~joanne/unixRC.pdf}{UNIX (shell) reference card}
\item \href{http://web.mit.edu/merolish/Public/vi-ref.pdf}{Vi refference card}
\item \href{https://www.gnu.org/software/emacs/refcards/pdf/refcard.pdf}{Emacs refference card}
\end{itemize}

Las dos últimas referencias se centran en los editores \textit{VI} y
\textit{Emacs}, disponibles en terminales UNIX y también en entornos
gráficos de ventanas, con menús, botones etc (especialmente,
\textit{Emacs}). \textit{VI} (o \textit{VIM}, una versión «mejorada»)
es especialmente ubicuo en sistemas UNIX.

En la literatura existe bastante documentación sobre el manejo de la
\textit{shell}, abarcando desde el manejo básico de la consola hasta
la programación de \textit{scripts}. Aquí citamos por ejemplo dos
libros de la editorial O'Reilly~\cite{Newham-bash,Albing-Vosen-Bash},
que, en versión PDF, están disponibles en internet%
\footnote{Los libros se pueden descargar en formato PDF,
  respectivamente, de:
  \begin{itemize}
  \item \url{https://doc.lagout.org/operating\%20system\%20/linux/Learning\%20the\%20bash\%20Shell\%20-\%20Unix\%20Shell\%20Programming.pdf}
  \item \url{http://index-of.es/Programming/Bash/O\%27Reilly\%20bash\%20CookBook.pdf}
  \end{itemize}
  }.


%%% Local Variables:
%%% mode: latex
%%% TeX-master: "efavanzados.tex"
%%% End:
