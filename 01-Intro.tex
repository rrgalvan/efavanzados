
%==================================================
\section{Introducción}
%==================================================

\begin{contenidos}
  Introducción a la «shell» en entornos Unix. Sistemas
  de control de versiones. Lenguaje de programación
  Python. Introducción al entorno de elementos finitos
  FEniCS.
\end{contenidos}

\subsection{Introducción a la «shell» en entornos Unix}
\label{sec:intr-la-shell}

El objetivo de esta sección es familiarizarse con: algunos de los
conceptos básicos de la \textit{shell} UNIX, que serán útiles para el
resto de los temas. En concreto:
\begin{itemize}
\item Órdenes básicas de la \textit{shell}
\item Introducción a la programación de \textit{scripts}
\end{itemize}

Nos centramos en la \textit{shell Bash}, la más extendida\footnote{Por
  supuesto existen muchas otras \textit{shells} UNIX, entre ellas
  destacamos a \href{https://www.zsh.org/}{Zsh}} en entornos
UNIX. Como punto de partida, se puede utilizar alguna de las numerosas
guías o «reference cards» disponibles en internet, por ejemplo:
\begin{itemize}
\item \href{https://www.cs.jhu.edu/~joanne/unixRC.pdf}{UNIX (shell) reference card}
\item \href{http://web.mit.edu/merolish/Public/vi-ref.pdf}{Vi refference card}
\item \href{https://www.gnu.org/software/emacs/refcards/pdf/refcard.pdf}{Emacs refference card}
\end{itemize}

Las dos últimas referencias se centran en los editores \textit{VI} y
\textit{Emacs}, disponibles en terminales UNIX y también en entornos
gráficos de ventanas, con menús, botones etc (especialmente,
\textit{Emacs}). \textit{VI} (o \textit{VIM}, una versión «mejorada»)
es especialmente ubicuo en sistemas UNIX.

En la literatura existe bastante documentación sobre el manejo de la
\textit{shell}, abarcando desde el manejo básico de la consola hasta
la programación de \textit{scripts}. Aquí citamos por ejemplo dos
libros de la editorial O'Reilly~\cite{Newham-bash,Albing-Vosen-Bash},
que, en versión PDF, están disponibles en internet%
\footnote{Los libros se pueden descargar en formato PDF,
  respectivamente, de:
  \begin{itemize}
  \item \url{https://doc.lagout.org/operating\%20system\%20/linux/Learning\%20the\%20bash\%20Shell\%20-\%20Unix\%20Shell\%20Programming.pdf}
  \item \url{http://index-of.es/Programming/Bash/O\%27Reilly\%20bash\%20CookBook.pdf}
  \end{itemize}
  }.

\subsection{Sistemas de control de versiones}
Los sistemas de control de versiones permiten gestionar los cambios que se realizan en ficheros (usualmente, código fuente de programas, documentos \LaTeX, etc.). Este tipo de sistemas está diseñado para almacenar los ficheros, mantener un histórico de las acciones realizadas sobre ellos y facilitar la colaboración de distintas personas en su edición. En la actualidad, el más utilizado es el sistema de control de versiones \href{https://git-scm.com/}{Git}. Los ficheros y los cambios realizados se pueden almacenar en distintos \textit{repositorios}, locales o remotos. Existen algunos sitios web para almacenar repositorios remotos, públicos o privados: \href{https://github.com}{Github}, \href{https://gitlab.com/}{Gitlab}, \href{https://bitbucket.org}{Bitbucket}...

Como introducción y punto de entrada a Git se recomienda seguir alguno de los tutoriales disponibles en internet, por ejemplo \url{https://swcarpentry.github.io/git-novice/}. De nuevo, entre la abundante literatura sobre el tema, destacamos aquí una referencia de la que podemos descargar una versión en formato PDF\footnote{\url{https://gitlab.atica.um.es/palazon.um.es/gitParaTodos/blob/50ade3c49f78bd372e0237530c9abf27afb5dd0d/bib/O\%27Reilly\%20Git\%20Pocket\%20Guide\%202013.pdf}}.

\subsection{Lenguaje de programación Python}
\label{sec:python}


Python es un un lenguaje de programación\ldots{}
\begin{itemize}
\item De alto nivel\footnote{``Cercano al lenguaje
    natural'' (en este caso, al inglés), a diferencia de
    ``lenguajes máquina''.}.
\item De propósito general\footnote{En el sentido de estar orientado a
    su uso para numerosos fines: desde programación de aplicaciones para
    sistemas operativos hasta, juegos, proceso de audio, cálculo
    científico, móviles\ldots{}}.
\item Interpretado\footnote{Es decir, existe un programa (el
    \emph{intérprete}) que lee y ejecuta los programas sin necesidad
    de un proceso previo de compilación.}. Como tal, tiene sus ventajas
  (se agiliza el ciclo de desarrollo de software) e inconvenientes
  (menos rápido que los lenguajes compilados).
\item Orientado al \emph{desarrollo rápido} y a la legibilidad de los programas.
\item Cuenta con extensiones (bibliotecas) que lo convierten en una
      potente herramienta en el \emph{cálculo numérico} y científico.
    \item Muy \emph{popular} entre los desarrolladores de software, en
      general. En la actualidad, ocupa la primera plaza en numerosas
      clasificaciones\footnote{Basta echar un vistazo a \href{https://www.google.com/search?channel=fs&client=ubuntu&q=lenguajes+de+ordenador+m\%C3\%A1s+usados}{Google: lenguajes de ordenador más usados}}.
    \item Por último, pero : con \emph{licencia libre},
  disponible en las arquitecturas habituales \emph{Windows},
  \emph{Mac}, \emph{Linux}.
\end{itemize}

Python abre un gran abanico de posibilidades para quienes estén
interesados la informática, el cálculo numérico y científico. En este
curso se presuponen conocimientos mínimos en este lenguaje, así como
en editores (como \textit{Atom}, \textit{Spider}, o \textit{Emacs}),
entornos de desarrollo (como \textit{Jupyter}) y bibliotecas
matemáticas (\textit{numpy}, \textit{matplotlib}...)
especializados. De nuevo, existe una numerosa bibliografía y recursos
disponibles en internet.

%%% Local Variables:
%%% mode: latex
%%% TeX-master: "efavanzados.tex"
%%% End:
