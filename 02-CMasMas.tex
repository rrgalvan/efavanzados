
%==================================================
\section{El lenguaje C++}
%==================================================

\begin{contenidos}
  El lenguaje de programación C++. Programación orientada a
  objetos. Implementación de bibliotecas de elementos finitos.
\end{contenidos}


\subsection{El lenguaje de programación C++}
\label{sec:el-lenguaje-c++}

Según \href{https://es.wikipedia.org/wiki/C\%2B\%2B}{Wikipedia}:
\begin{quote}\it
  C++ es un lenguaje de programación diseñado en 1979 por Bjarne Stroustrup. La intención de su creación fue extender al lenguaje de programación C mecanismos que permiten la manipulación de objetos. En ese sentido, desde el punto de vista de los lenguajes orientados a objetos, C++ es un lenguaje híbrido.

Posteriormente se añadieron facilidades de programación genérica, que se sumaron a los paradigmas de programación estructurada y programación orientada a objetos. Por esto se suele decir que el C++ es un lenguaje de programación multiparadigma.

Actualmente existe un estándar, denominado ISO C++, al que se han adherido la mayoría de los fabricantes de compiladores más modernos.
\end{quote}

El estándar ISO C++ ha evolucionado con los años y, a pesar de que se trató de no hacer cambios importantes en el núcleo del lenguaje, la versión publicada en el año 2011 marcó diferencias entre las versiones anteriores ``clásicas'' y versiones de C++ ``modernas''. De nuevo de Wikipedia:

\begin{quote}\it
  Entre las características del nuevo estándar [C++11] se pueden destacar:
  \begin{itemize}
  \item Funciones lambda;
  \item Referencias rvalue;
  \item La palabra reservada auto;
  \item Inicialización uniforme;
  \item Plantillas con número variable de argumentos.
  \end{itemize}
  Además se ha actualizado la biblioteca estándar del lenguaje.
\end{quote}
Para introducirse en el lenguaje se suele considerar buena costumbre comenzar estudiando C++ ``clásico'' para a continuación sumergirse en los cambios que introdujo C++11 (y versiones posteriores: C++14, C++17,...).

\paragraph{Algunos libros/bibliografía sobre C++ ``clásico''}
\begin{enumerate}

\item \href{http://www-apr.lip6.fr/~tierny/stuff/teaching/tierny_visualization_introductionC++.pdf}{J. Tierny -- A Quick Introduction to C++ Programming}. Introducción breve (26 páginas en PDF). Se incluyen ficheros \textit{CMake} para la compilación de los ejemplos.
\item \href{https://homes.cs.washington.edu/~tom/c++example/}{Thomas Anderson -- A Quick Introduction to C++}. Introducción rápida. En la web se incluye el código fuente de los ejemplos. Pero un poco anticuada, al menos la sección ``5 Features To Avoid Like the Plague'' es polémica (¿evitar la biblioteca estándar? ¿la clase \texttt{vector}?.
\item Bruce Eckel, Chuck Allison -- Thinking C++, \href{http://vergil.chemistry.gatech.edu/resources/programming/pdf/TIC2Vone.pdf}{volumen 1} y \href{https://www.cs.rit.edu/~cs4/Thinking-in-C++2nd-ed-Volume-2.pdf}{volumen 2}.
\item \href{https://github.com/manish-old/ebooks-2/blob/master/O'Reilly\%20-\%20Practical\%20C\%2B\%2B\%20Programming.pdf}
{Libro ed. O'Reilly -- Practical C++ Programming}
\end{enumerate}

\paragraph{Algunos libros/bibliografía sobre C++ ``moderno''}
\begin{enumerate}
\item \href{https://consense.com.ua/en/lib/book/cpp_modern_cpp_progr_cookbook}{Marius Bancila.
"Modern C++ Programming Cookbook"}
\item \href{https://changkun.de/modern-cpp/pdf/modern-cpp-tutorial-en-us.pdf}{Changkun Ou. ``Modern C++ Tutorial: C++11/14/17/20 On the Fly''}

\end{enumerate}


\section{Algunas líneas interesantes en las que profundizar}
\label{sec:01:profundizar}


\begin{itemize}
\item ¿Cómo ejecutar software C/C++ desde Python?
\item \href{http://hplgit.github.io/fenics-mixed/doc/pub/fenics-mixed.html}{¿Cómo combinar FEniCS con otro software? (Fortran, Octave/Matlab...)}
\end{itemize}



%%% Local Variables:
%%% mode: latex
%%% TeX-master: "efavanzados.tex"
%%% End:
