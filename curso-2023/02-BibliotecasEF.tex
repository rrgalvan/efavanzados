%==================================================
\section{Bibliotecas de Elementos Finitos}
%==================================================

\begin{contenidos}
Implemenación de Bibliotecas de elementos finitos.
La biblioteca Julia Gridap. La biblioteca Python/C++ FEniCS.
\end{contenidos}


\subsection{Introducción al entorno de elementos finitos FEniCS}
%---------------------------------------------------------------

\label{sec:FEniCS}
La mejor descripción de \textit{FEniCS} la encontramos en su \href{https://fenicsproject.org/}{página web}:
\begin{quote}
  \it FEniCS is a popular open-source (LGPLv3) computing platform for solving partial differential equations (PDEs). FEniCS enables users to quickly translate scientific models into efficient finite element code. With the high-level Python and C++ interfaces to FEniCS, it is easy to get started, but FEniCS offers also powerful capabilities for more experienced programmers. FEniCS runs on a multitude of platforms ranging from laptops to high-performance clusters.
\end{quote}

En los últimos años, el
\href{https://fenicsproject.org/download/}{proceso de instalación} se
ha simplificado y gracias a ello FEniCS es hoy una de las principales
referencias en la resolución numérica de EDP mediante el Método de los
Elementos Finitos.

Como documentación, es interesante destacar
\href{https://fenicsproject.org/book/}{The FEniCS Book}, un libro de
más de 700 pátinas que incluye ejemplos avanzados y aplicaciones en
numerosos campos (fluidos, elasticidad, biología...) que se puede
descargar en formato PDF.

\lstinputlisting[language=python]{src/FEniCS-first-example.py}

\subsection{Algunas líneas interesantes en las que profundizar}
\label{sec:01:profundizar}

\begin{itemize}
\item ¿Cómo funciona la interfaz C++ de \textit{FEniCS}? ¿Es más rápida que la interfaz Python? \url{https://fenicsproject.org/pub/course/lectures/2017-nordic-phdcourse/lecture_18_cpp_fenics.pdf}

\item Para compilar en C++, es conveniente tener nociones de las herramientas \textit{GNU Make} y \textit{CMake}.
  \begin{itemize}
  \item \url{https://rodin.uca.es/xmlui/handle/10498/13276}
  \item \url{https://cmake.org/cmake/help/v3.20/guide/tutorial/index.html}
  \end{itemize}
\end{itemize}


%%% Local Variables:
%%% mode: latex
%%% TeX-master: "menuavedp.tex"
%%% End:
