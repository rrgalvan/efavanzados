
\section{Implementación en grandes ordenadores de métodos numéricos paralelos en espacio y tiempo}

\subsection{El supercomputador de la UCA}
\label{sec:supercomputador-UCA}
La UCA dispone del superordenador \textit{CAI} (Cluster de Apoyo a la
Investigación).  La forma de acceder al mismo, la política para lanzar
trabajos y otras cuestiones básicas se resumen en
\url{https://supercomputacion.uca.es/recursos/documentacion/preguntas-mas-frecuentes}

Al acceder, el usuario está situado en el nodo de entrada, donde puede
realizar pruebas, copiar ficheros, etc. y enviar programas para ser
ejecutados en los nodos de cálculo.

Se cuenta con una cantidad limitada de disco duro en la partición
\texttt{/home} y más disco en la partición \texttt{/scratch\_local}
(\url{https://supercomputacion.uca.es/limites/}).

\subsection{Implementación en grandes ordenadores de métodos numéricos paralelos en espacio y tiempo}

\subsection{El superordenador de la UCA}

\subsubsection*{Hardware}
El \textbf{hardware} se describe en
\url{https://supercomputacion.uca.es/cluster-de-supercomputacion/hardware/},
en la actualidad:
\begin{itemize}
\item 48 nodos, cada uno de los cuales dispone de 2 procesadores Intel Xeon E5 2670 a 2,6 GHz.
\item Estos procesadores cuentan con 8 núcleos, por tanto cada nodo dispone $8\times 2=16$ núcleos
\item En total, disponemos de $48\times 16=768$ núcleos de cálculo
\end{itemize}

\subsubsection*{Software}
El \textbf{software} disponible se describe en \url{https://supercomputacion.uca.es/software/}
\begin{itemize}
\item Muchas de los paquetes de software disponibles sólo están
  disponibles como \textbf{módulos}
  \begin{itemize}
  \item Deben cargarse antes de ser utilizados, usando
    \texttt{module load}
  \item Ejemplo:
    \begin{lstlisting}[language=sh]
      module load FEniCS
    \end{lstlisting}
  \item Pulsando \texttt{module load} \texttt{<tabulador>} aparecen
    todos los módulos disponibles
  \item \texttt{moudule list}: lista módulos cargados
  \item Usar \texttt{module unload} para eliminar un módulo concreto

  \item Más información: \texttt{module} \texttt{<tabulador>} o \texttt{man module}
  \end{itemize}
\end{itemize}

\subsection{El sistema de colas \textit{slurm} para lanzar trabajos}
\label{sec:supercomputador-UCA}
Las tareas que envía cada usuario deben esperar, antes de su
ejecución, a que el sistema le asigne los recursos necesarios (CPU,
memoria, ...). El documento
«\href{http://supercomputacion.uca.es/recursos/documentacion/politicas-de-gestion-de-colas}{Política
  de Gestión de Colas}», resume la política de gestión de recursos que
se utiliza para evitar que un usuario colapsase el supercomputador y
que el uso de los recursos sea más equitativo entre todos los
usuarios.

El funcionamiento básico del sistema de colas
\href{https://slurm.schedmd.com/quickstart.html}{slurm} se puede consultar en su
web o en la página de documentación de supercomputación
\url{https://supercomputacion.uca.es/recursos/documentacion}.

Un resumen, extraído de su web:
\begin{itemize}
\item \texttt{sinfo}: reports the state of partitions and nodes
  managed by Slurm.
\item \texttt{squeue}: reports the state of jobs. By default, it
  reports the running jobs in priority order and then the pending jobs
  in priority order.
\item \texttt{sbatch} is used to submit a job script for later
  execution.
\item \texttt{srun} is used to submit a job for execution or
  initiate job steps in real time.
\item \texttt{scancel} is used to cancel a pending or running job or
  job step.
\end{itemize}

La orden usual para lanzar trabajos es \textbf{sbatch}. Se le debe
pasar un \textit{script} de tipo \textit{shell} en el que se
configuran los detalles del proceso que será ejecutado.
\begin{itemize}
\item El programa
será cargado en el sistema de colas, en estado \textit{PENDING}.
\item Cuando haya recursos disponibles, pasará a estado
  \textit{RUNNING} y será ejecutado.
\item Al finalizar, la salida se graba en un archivo (por defecto, en
  el mismo directorio que el \textit{script}), así como la salida de
  errores.
\end{itemize}
Para \textbf{generar el \textit{script}} para el sistema de colas, se aconseja:
\begin{itemize}
\item Utilizar el \textit{Generador de scripts} que está disponible en
https://cai.uca.es/supercomputacion/usuario.php (enlace arriba a la izquierda)
\item O bien utilizar un borrador previo, como el que aparece a continuación (tomado de \url{https://supercomputacion.uca.es/recursos/documentacion/preguntas-mas-frecuentes}.
\end{itemize}

\subsubsection*{Borrador para el sistema de colas \textit{slurm}}
    \begin{lstlisting}[language=sh]
#!/bin/bash
#------- DIRECTIVAS SBATCH -------
# Datos genericos aplicables a todos los trabajos:
#SBATCH --partition=cn
# - #SBATCH --exclusive
#SBATCH --mail-user=SU.CORREO@uca.es
#SBATCH --mail-type=BEGIN,END,FAIL,TIME_LIMIT_90
#SBATCH --requeue
#SBATCH --share
# Esto para salidas no controladas:
#SBATCH --error=/home/GRUPO/USUARIO/job.%J.err
#SBATCH --output=/home/GRUPO/USUARIO/job.%J.out
# - #SBATCH --workdir="/scratch/USUARIO"
# - #SBATCH --workdir="/scratch_local/USUARIO"
#SBATCH --workdir="/home/GRUPO/USUARIO"
# Descripcion del trabajo:
#SBATCH --job-name="TEST"
#SBATCH --comment="Prueba de SBATCH"
# *** MUY IMPORTANTE ***
# Parametrizaci'on del trabajo
# - #SBATCH --account=CUENTA
#SBATCH --tasks=1
# - #SBATCH --cpus-per-task=1
# - #SBATCH --nodes=1
# - #SBATCH --tasks-per-node=1
#SBATCH --time=0-00:05:00
#SBATCH --mem=1GB
# - #SBATCH --gres=gpu:tesla:2
#------- CONFIGURACION ENTORNO -------
# Variables de ambiente exportadas para que est'en disponibles en
# todos los procesos hijo
export DATASOURCE=$SLURM_SUBMIT_DIR/input
export DATAEND=$SLURM_SUBMIT_DIR/output
export SCRATCH1=/scratch/$USER
export SCRATCH2=/scratch_local/$USER
# Carga de m'odulos
#module load matlab
# Configuraci'on del scratch
mkdir -p $SCRATCH1
#------- COPIA DE DATOS AL SCRATCH -------
#sbcast --force --fanout=8 --size=100m $DATASOURCE/$SLURM_JOB_NAME.in
$SCRATCH2/$SLURM_JOB_NAME.in
#------- EJECUTAMOS EL PROGRAMA -------
srun miprograma < $DATASOURCE/$SLURM_JOB_NAME.in > $SCRATCH1/$SLURM_JOB_NAME.out
#srun miprograma < $SCRATCH2/$SLURM_JOB_NAME.in > $SCRATCH1/$SLURM_JOB_NAME.out
RESULT=$?
#------- SALVAMOS LOS RESULTADOS -------
mv $SCRATCH1 $DATAEND/$SLURM_JOB_ID
#------- ELIMINAMOS FICHEROS TEMPORALES -------
rm -rf $SCRATCH1 $SCRATCH2
#------- FIN -------
exit $RESULT
\end{lstlisting}

%%% Local Variables:
%%% mode: latex
%%% TeX-master: "efavanzados.tex"
%%% End:
