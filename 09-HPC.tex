
%==================================================
\section{Grandes ordenadores y paralelización}
%==================================================

\begin{contenidos}
Implementación en grandes ordenadores de métodos numéricos paralelos en espacio y tiempo.
\end{contenidos}

\subsection{El supercomputador de la UCA}
\label{sec:supercomputador-UCA}
La UCA dispone del superordenador \textit{CAI} (Cluster de Apoyo a la
Investigación).  La forma de acceder al mismo, la política para lanzar
trabajos y otras cuestiones básicas se resumen en
\url{https://supercomputacion.uca.es/recursos/documentacion/preguntas-mas-frecuentes}

Al acceder, el usuario está situado en el nodo de entrada, donde puede
realizar pruebas, copiar ficheros, etc. y enviar programas para ser
ejecutados en los nodos de cálculo.

Se cuenta con una cantidad limitada de disco duro en la partición
\texttt{/home} y más disco en la partición \texttt{/scratch\_local}
(\url{https://supercomputacion.uca.es/limites/}).

\subsection{Implementación en grandes ordenadores de métodos numéricos paralelos en espacio y tiempo}

\subsection{El superordenador de la UCA}

\subsubsection*{Hardware}
El \textbf{hardware} se describe en
\url{https://supercomputacion.uca.es/cluster-de-supercomputacion/hardware/},
en la actualidad:
\begin{itemize}
\item 48 nodos, cada uno de los cuales dispone de 2 procesadores Intel Xeon E5 2670 a 2,6 GHz.
\item Estos procesadores cuentan con 8 núcleos, por tanto cada nodo dispone $8\times 2=16$ núcleos
\item En total, disponemos de $48\times 16=768$ núcleos de cálculo
\end{itemize}

\subsubsection*{Software}
El \textbf{software} disponible se describe en \url{https://supercomputacion.uca.es/software/}
\begin{itemize}
\item Muchas de los paquetes de software disponibles sólo están
  disponibles como \textbf{módulos}
  \begin{itemize}
  \item Deben cargarse antes de ser utilizados, usando
    \texttt{module load}
  \item Ejemplo (para usar FEniCS):
    \begin{lstlisting}[language=sh]
      module load anaconda/3.7
      conda activate fenicsproject # Entramos en el entorno fenicsproject en el que esta instalado FEniCS con Anaconda
      python programa.py # Ejecutamos el programa que carga el modulo de FEniCS mediante ``import fenics''
      conda deactivate # Ejecutamos esta orden para salir del entorno fenicsproject
    \end{lstlisting}
  \item Pulsando \texttt{module load} \texttt{<tabulador>} aparecen
    todos los módulos disponibles
  \item \texttt{moudule list}: lista módulos cargados
  \item Usar \texttt{module unload} para eliminar un módulo concreto

  \item Más información: \texttt{module} \texttt{<tabulador>} o \texttt{man module}
  \end{itemize}
\end{itemize}

\subsection{El sistema de colas \textit{slurm} para lanzar trabajos}
\label{sec:slurm}
Las tareas que envía cada usuario deben esperar, antes de su
ejecución, a que el sistema le asigne los recursos necesarios (CPU,
memoria, ...). El documento
«\href{http://supercomputacion.uca.es/recursos/documentacion/politicas-de-gestion-de-colas}{Política
  de Gestión de Colas}», resume la política de gestión de recursos que
se utiliza para evitar que un usuario colapsase el supercomputador y
que el uso de los recursos sea más equitativo entre todos los
usuarios.

El funcionamiento básico del sistema de colas
\href{https://slurm.schedmd.com/quickstart.html}{slurm} se puede consultar en su
web o en la página de documentación de supercomputación
\url{https://supercomputacion.uca.es/recursos/documentacion}.

Un resumen, extraído de su web:
\begin{itemize}
\item \texttt{sinfo}: reports the state of partitions and nodes
  managed by Slurm.
\item \texttt{squeue}: reports the state of jobs. By default, it
  reports the running jobs in priority order and then the pending jobs
  in priority order.
\item \texttt{sbatch} is used to submit a job script for later
  execution.
\item \texttt{srun} is used to submit a job for execution or
  initiate job steps in real time.
\item \texttt{scancel} is used to cancel a pending or running job or
  job step.
\end{itemize}

La orden usual para lanzar trabajos es \textbf{sbatch}. Se le debe
pasar un \textit{script} de tipo \textit{shell} en el que se
configuran los detalles del proceso que será ejecutado.
\begin{itemize}
\item El programa
será cargado en el sistema de colas, en estado \textit{PENDING}.
\item Cuando haya recursos disponibles, pasará a estado
  \textit{RUNNING} y será ejecutado.
\item Al finalizar, la salida se graba en un archivo (por defecto, en
  el mismo directorio que el \textit{script}), así como la salida de
  errores.
\end{itemize}
Para \textbf{generar el \textit{script}} para el sistema de colas, se aconseja:
\begin{itemize}
\item Utilizar el \textit{Generador de scripts} que está disponible en
https://cai.uca.es/supercomputacion/usuario.php (enlace arriba a la izquierda)
\item O bien utilizar un borrador previo, como el que aparece a continuación (tomado de \url{https://supercomputacion.uca.es/recursos/documentacion/preguntas-mas-frecuentes}.
\end{itemize}

\subsubsection*{Borrador para el sistema de colas \textit{slurm}}
    \begin{lstlisting}[language=sh]
#!/bin/bash
#------- DIRECTIVAS SBATCH -------
# Datos genericos aplicables a todos los trabajos:
#SBATCH --partition=cn
# - #SBATCH --exclusive
#SBATCH --mail-user=SU.CORREO@uca.es
#SBATCH --mail-type=BEGIN,END,FAIL,TIME_LIMIT_90
#SBATCH --requeue
#SBATCH --share
# Esto para salidas no controladas:
#SBATCH --error=/home/GRUPO/USUARIO/job.%J.err
#SBATCH --output=/home/GRUPO/USUARIO/job.%J.out
# - #SBATCH --workdir="/scratch/USUARIO"
# - #SBATCH --workdir="/scratch_local/USUARIO"
#SBATCH --workdir="/home/GRUPO/USUARIO"
# Descripcion del trabajo:
#SBATCH --job-name="TEST"
#SBATCH --comment="Prueba de SBATCH"
# *** MUY IMPORTANTE ***
# Parametrizaci'on del trabajo
# - #SBATCH --account=CUENTA
#SBATCH --tasks=1
# - #SBATCH --cpus-per-task=1
# - #SBATCH --nodes=1
# - #SBATCH --tasks-per-node=1
#SBATCH --time=0-00:05:00
#SBATCH --mem=1GB
# - #SBATCH --gres=gpu:tesla:2
#------- CONFIGURACION ENTORNO -------
# Variables de ambiente exportadas para que est'en disponibles en
# todos los procesos hijo
export DATASOURCE=$SLURM_SUBMIT_DIR/input
export DATAEND=$SLURM_SUBMIT_DIR/output
export SCRATCH1=/scratch/$USER
export SCRATCH2=/scratch_local/$USER
# Carga de m'odulos
#module load matlab
# Configuraci'on del scratch
mkdir -p $SCRATCH1
#------- COPIA DE DATOS AL SCRATCH -------
#sbcast --force --fanout=8 --size=100m $DATASOURCE/$SLURM_JOB_NAME.in
$SCRATCH2/$SLURM_JOB_NAME.in
#------- EJECUTAMOS EL PROGRAMA -------
srun miprograma < $DATASOURCE/$SLURM_JOB_NAME.in > $SCRATCH1/$SLURM_JOB_NAME.out
#srun miprograma < $SCRATCH2/$SLURM_JOB_NAME.in > $SCRATCH1/$SLURM_JOB_NAME.out
RESULT=$?
#------- SALVAMOS LOS RESULTADOS -------
mv $SCRATCH1 $DATAEND/$SLURM_JOB_ID
#------- ELIMINAMOS FICHEROS TEMPORALES -------
rm -rf $SCRATCH1 $SCRATCH2
#------- FIN -------
exit $RESULT
\end{lstlisting}

\subsection*{Paralelización de programa FEniCS}

Por defecto, si a un programa FEniCS (\textit{programa.py}) le indicamos lo siguiente, realizará paralelización del código entre los procesadores con memoria compartida\footnote{Esto puede que no sea necesario, FEniCS trabaja en paralelo con memoria compartida usando PETSc por defecto como backend.}:
\begin{lstlisting}[language=python]
	from dolfin import *
	
	parameters['linear_algebra_backend'] = 'PETSc'
	parameters["mesh_partitioner"] = "ParMETIS" # "SCOTCH"
	
	...
\end{lstlisting}


Para paralelizar el programa FEniCS (\textit{programa.py}) en el cluster de supercomputación de la UCA con memoria distribuida (usando varios nodos de cálculo) podemos usar además el siguiente script donde NODOS es el número de nodos del cluster que queremos utilizar:
\begin{lstlisting}[language=sh]
	#!/bin/bash
	
	# Before conda activate, see https://fenicsproject.discourse.group/t/fenics-from-conda-doesnt-import/3502/6
	
	# module load anaconda/3.7 # Podria ser necesario cargar el modulo si necesitamos actualizar las variables de tipo $PATH de nuestra shell con las que incorpora el modulo, por ejemplo, si hacemos una instalacion local de anaconda y queremos usar la instalacion sudo
	
	CONDA_DIR=/apps/anaconda3-python3.7 # Ruta del directorio de Anaconda
	source $CONDA_DIR/etc/profile.d/conda.sh # Ruta del script de Anaconda
	conda activate fenicsproject
	
	time mpirun -n NODOS python programa.py # De esta forma ejecutamos tantos procesos como indiquemos con NODOS, hay que cambiar el numero de procesos y ajustarlo al numero de nodos del cluster que queramos usar.
\end{lstlisting}

Ahora, ejecutamos el script anterior (\textit{script.sh}) con la siguiente orden:
\begin{lstlisting}[language=sh]
sbatch -n NODOS script.sh
\end{lstlisting}
Si usamos el sistema de colas \texttt{slurm}, no hace falta indicar el numero de tareas en \texttt{mpirun}, sino simplemente a \texttt{sbatch}.

\subsubsection*{Ejemplo con 1 nodo:}

Creamos el script para ejecutar el programa \textit{diffusion-simple.py}.
\begin{lstlisting}[language=sh]
	#!/bin/bash
	
	# Before conda activate, see https://fenicsproject.discourse.group/t/fenics-from-conda-doesnt-import/3502/6
	CONDA_DIR=/apps/anaconda3-python3.7 # Ruta del directorio de Anaconda
	source $CONDA_DIR/etc/profile.d/conda.sh # Ruta del script de Anaconda
	conda activate fenicsproject
	
	time mpirun python diffusion-simple.py
\end{lstlisting}

Ahora, ejecutamos el script anterior (\textit{diffus-sbatch.sh}) con la siguiente orden:
\begin{lstlisting}[language=sh]
	sbatch -n 1 diffus-sbatch.sh
\end{lstlisting}

\textit{Tiempo:} 0m25.700s.

\subsubsection*{Ejemplo con 2 nodos:}
Creamos el script para ejecutar el programa \textit{diffusion-simple.py}.
\begin{lstlisting}[language=sh]
	#!/bin/bash
	
	# Before conda activate, see https://fenicsproject.discourse.group/t/fenics-from-conda-doesnt-import/3502/6
	CONDA_DIR=/apps/anaconda3-python3.7 # Ruta del directorio de Anaconda
	source $CONDA_DIR/etc/profile.d/conda.sh # Ruta del script de Anaconda
	conda activate fenicsproject
	
	time mpirun python diffusion-simple.py
\end{lstlisting}

Ahora, ejecutamos el script anterior (\textit{diffus-sbatch.sh}) con la siguiente orden:
\begin{lstlisting}[language=sh]
	sbatch -n 2 diffus-sbatch.sh
\end{lstlisting}

\textit{Tiempo:} 0m26.850s.

\subsubsection*{Ejemplo con 4 nodos:}
Creamos el script para ejecutar el programa \textit{diffusion-simple.py}.
\begin{lstlisting}[language=sh]
	#!/bin/bash
	
	# Before conda activate, see https://fenicsproject.discourse.group/t/fenics-from-conda-doesnt-import/3502/6
	CONDA_DIR=/apps/anaconda3-python3.7 # Ruta del directorio de Anaconda
	source $CONDA_DIR/etc/profile.d/conda.sh # Ruta del script de Anaconda
	conda activate fenicsproject
	
	time mpirun python diffusion-simple.py
\end{lstlisting}

Ahora, ejecutamos el script anterior (\textit{diffus-sbatch.sh}) con la siguiente orden:
\begin{lstlisting}[language=sh]
	sbatch -n 4 --ntasks-per-node 1 diffus-sbatch.sh
\end{lstlisting}

\textit{Tiempo:} 0m21.582s.

\subsubsection*{Ejemplo con 8 nodos:}
Creamos el script para ejecutar el programa \textit{diffusion-simple.py}.
\begin{lstlisting}[language=sh]
	#!/bin/bash
	
	# Before conda activate, see https://fenicsproject.discourse.group/t/fenics-from-conda-doesnt-import/3502/6
	CONDA_DIR=/apps/anaconda3-python3.7 # Ruta del directorio de Anaconda
	source $CONDA_DIR/etc/profile.d/conda.sh # Ruta del script de Anaconda
	conda activate fenicsproject
	
	time mpirun python diffusion-simple.py
\end{lstlisting}

Ahora, ejecutamos el script anterior (\textit{diffus-sbatch.sh}) con la siguiente orden:
\begin{lstlisting}[language=sh]
	sbatch -n 8 --ntasks-per-node 1 diffus-sbatch.sh
\end{lstlisting}

\textit{Tiempo:} 0m18.326s.

\subsubsection*{Script más complejo para ejecutar FEniCS en el superordenador}
Para ejecutar FEniCS en el superordenador cai3 y que nos avise por correo de cuando comienza a ejecutarse y termina de ejecutarse un programa, podemos usar el siguiente script. Tenemos que tener cuidado con varias cosas:
\begin{itemize}
	\item Tenemos que crear un directorio llamado \textbf{out} dentro del directorio de trabajo.
	\item Si modificamos el nombre del programa en la variable \textbf{PROGRAM}, debemos cambiarlo también en \textbf{SBATCH --job-name} e indicar el \textbf{mismo nombre}.
	\item En el parámetro \textbf{SBATCH --nodes} indicamos el número de nodos en el que se ejecuta nuestro programa con memoria dsitribuida. La orden \texttt{mpirun} reconoce los nodos que le pedimos a \texttt{slurm}.
\end{itemize}

\begin{lstlisting}[language=sh]
	#!/bin/bash
	#SBATCH --job-name=diffusion-simple.py
	#SBATCH --partition=cn
	#SBATCH --ntasks=10
	#SBATCH --time=10-00:00:00
	#SBATCH --error=./out/job.%J.err
	#SBATCH --output=./out/job.%J.out
	#SBATCH --mail-user=daniel.acosta@uca.es
	#SBATCH --mail-type=ALL
	#SBATCH --requeue
	
	#------------------------------------
	# Global variables
	export PROGRAM=./diffusion-simple.py # Change this variable in #SBATCH --job-name too
	
	# REMEMBER TO MAKE A DIRECTORY CALLED out INSIDE THE WORK DIRECTORY TO SAVE THE OUTPUTS
	
	#------------------------------------------------------------------
	# Environment configuration
	
	# Exported variables so that they are available in all children jobs
	export WORKDIR=$SLURM_SUBMIT_DIR
	export PYTHON=python
	
	# module load anaconda/3.7
	
	# Before conda activate, see https://fenicsproject.discourse.group/t/fenics-from-conda-doesnt-import/3502/6
	CONDA_DIR=/apps/anaconda3-python3.7
	source $CONDA_DIR/etc/profile.d/conda.sh
	
	conda activate fenicsproject
	
	#------------------------------------------------------------------
	# Run pogram
	time mpirun $PYTHON $WORKDIR/$PROGRAM
	RESULT=$?
	
	#------------------------------------------------------------------
	# Save results, remove tmp files and exit
	# mv $SCRATCH1 $DATAEND/$SLURM_JOB_ID
	# rm -rf $SCRATCH1
	exit $RESULT
	
\end{lstlisting}

Para lanzar el script (supongamos que se llama \textit{script.sh}), ejecutamos el siguiente comando:
\begin{lstlisting}[language=sh]
	sbatch script.sh
\end{lstlisting}

%%% Local Variables:
%%% mode: latex
%%% TeX-master: "efavanzados.tex"
%%% End:
